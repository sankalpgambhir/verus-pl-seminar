
\documentclass[
  aspectratio=1610, 
  xcolor={dvipsnames},
  % handout
]{beamer}

\usetheme{metropolis}

\setbeamercolor{background canvas}{bg = white}

% macro.tex

\usepackage{graphicx}
\usepackage{amsmath}
\usepackage{multicol}
\usepackage{tabularx}
\usepackage{array}
\usepackage{forest}
\newcolumntype{x}[1]{>{\centering\arraybackslash\hspace{0pt}}p{#1}}
% \setlength{\columnseprule}{1pt}
% \def\columnseprulecolor{\color{block body.fg}}
\usepackage{tikz, pgfplots}
\usetikzlibrary{shapes.geometric, shapes, fit, arrows, calc, fit, positioning, chains, arrows.meta, backgrounds, calc}

% subject specific
\usepackage{amsmath, amssymb}
\usepackage{bussproofs}

% drawing
\usepackage{pstricks}
\usepackage{tikz, tkz-orm, pgfplots}
\usetikzlibrary{arrows,shapes,backgrounds,decorations.markings}
\usetikzlibrary{matrix,positioning,decorations.pathreplacing,calc,tikzmark}
\pgfplotsset{width=7.5cm,compat=1.16}
\usepgfplotslibrary{fillbetween}

% listings
\usepackage[final]{listings}
% \usepackage{unicode-math}
% \usepackage{xunicode}
% \usepackage{fontspec}
% \setmonofont[
%   Contextuals={Alternate}
% ]{Fira Code}

\definecolor{codegreen}{rgb}{0,0.6,0}
\definecolor{codegray}{rgb}{0.5,0.5,0.5}
\definecolor{codepurple}{rgb}{0.58,0,0.82}
\definecolor{backcolour}{rgb}{1,1,1}

\lstdefinestyle{stolenstyle}{
    backgroundcolor=\color{backcolour},
    commentstyle=\color{codegreen},
    keywordstyle=\color{magenta},
    numberstyle=\tiny\color{codegray},
    stringstyle=\color{codepurple},
    basicstyle=\ttfamily\footnotesize,
    breakatwhitespace=false,
    breaklines=true,
    captionpos=b,
    keepspaces=true,
    numbers=left,
    numbersep=5pt,
    showspaces=false,
    showstringspaces=false,
    showtabs=false,
    tabsize=2,
    inputencoding=utf8,
    extendedchars=true,
    language=scala,
    % texcl=false,
    mathescape=false,
    % escapechar=\&,
    literate = 
        {=>}{{=>}} {2}
        {->}{{->}} {2}
        {_}{{\_}} {1}
        {|}{\textbar} {1}
        {==>}{{==>}} {3},
}

\lstdefinelanguage{rust}{
    keywords={abstract,alignof,as,become,box,break,const,continue,crate,do,else,enum,extern,false,final,fn,for,if,impl,in,let,loop,macro,match,mod,move,mut,offsetof,override,priv,proc,pub,pure,ref,return,self,Self,sizeof,static,struct,super,trait,true,type,typeof,unsafe,unsized,use,virtual,where,while,yield,ghost,tracked,requires,ensures,old},
    morecomment=[l]{//},
    morecomment=[s]{/*}{*/},
    morestring=[b]",
    morestring=[b]',
    sensitive=true,
}

\lstset{style=stolenstyle}

\lstdefinestyle{lisa}{
    backgroundcolor=\color{backcolour},   
    commentstyle=\color{codegreen},
    keywordstyle=\color{magenta},
    numberstyle=\tiny\color{codegray},
    stringstyle=\color{codepurple},
    basicstyle=\linespread{1.4}\ttfamily\footnotesize,
    breakatwhitespace=false,         
    breaklines=false,                 
    captionpos=b,                    
    keepspaces=true,                 
    numbers=left,                    
    numbersep=5pt,                  
    showspaces=false,                
    showstringspaces=false,
    showtabs=false,                  
    tabsize=2,
    keywordstyle = [2]{\color{OliveGreen}},
    keywordstyle = [3]{\color{Blue}},
    keywordstyle = [4]{\color{BrickRed}},
    morekeywords = [2]{val, Theorem, have, thenHave, andThen, by},
    morekeywords = [3]{AccOutNil},
    morekeywords = [4]{Apply},
    literate = 
        {===}{{===}} 3
        {::}{{::}}   2
        {->}{{->}}   2,
}

% \lstset{style=lisa}

% links and citations
\usepackage{hyperref, bookmark}
\hypersetup{hidelinks} % hide boxes around links ew
% pls add color
\usepackage{csquotes}
% \usepackage{polyglossia}
% \setdefaultlanguage{english}
\usepackage[
    style=numeric,
    sorting=none,
    maxnames=99,
    language=english
]{biblatex}
\addbibresource{biblio.bib}

\usepackage{mathpartir}
\usepackage{stmaryrd}

\usepackage{pifont}
\newcommand{\cmark}{\ding{51}}%
\newcommand{\xmark}{\ding{55}}%

\newcommand{\mathcomment}[1]{\text{\textcolor{black!50}{\emph{#1}}}}


\title{Verus: Verifying Rust Programs Using Linear Ghost Types}
\subtitle{Andrea Lattuada, Travis Hance, Chanhee Cho, Matthias Brun, Isitha Subasinghe, Yi Zhou, Jon Howell, Bryan Parno, Chris Hawblitzel}
\author{Sankalp Gambhir}

\begin{document}

\maketitle

\begin{frame}[fragile]
  \frametitle{Rust and Ownership}

  Rust has ownership and borrowing built into its type system, and it appears in
  a few different forms to the user: 

  \pause

  \begin{center}
    \begin{tabular}{c c c}
      \textbf{Instance}     & \textbf{Mutability} & \textbf{Semantics}                       \\
      \pause
      \lstinline|x: T|      & \xmark              & Affine                                   \\
      \pause
      \lstinline|mut x: T|  & \cmark              & Affine                                   \\
      \pause
      \lstinline|x: &T|     & \xmark              & Shared with unrestricted (immutable) use \\
      \pause
      \lstinline|x: &mut T| & \cmark              & Exclusive reference, affine              \\
    \end{tabular}
  \end{center}

  \pause
  % there are a few other options
  Others: \lstinline|&a' T|, \lstinline|&mut a' T|, \lstinline|*const T|,
  \lstinline|*mut T|, \lstinline|Box<T>|, \lstinline|Arc<T>|,
  \lstinline|Mutex<T>|

  % but we'll talk about only a couple of them, as and when they show up

\end{frame}

\begin{frame}[fragile]
  \frametitle{Pop Quiz}

  Which of the following Rust programs should compile?

\end{frame}

\begin{frame}[fragile]

  \begin{lstlisting}[language=rust]
fn main() {
    let mut x = 5;
    let r1 = &mut x;
    let r2 = &mut x;
    println!("{} {}", r1, r2);
}
\end{lstlisting}

  \pause {\centering \xmark}

\end{frame}

\begin{frame}[fragile]

  \begin{lstlisting}[language=rust]
fn main() {
    let s = String::from("hello");
    let s2 = s;
    println!("{}", s);
}
\end{lstlisting}

  \pause {\centering \xmark}

\end{frame}

\begin{frame}[fragile]

  \begin{lstlisting}[language=rust]
fn say_what() -> &String {
    let s = String::from("hello");
    &s
}
\end{lstlisting}

  \pause {\centering \xmark}

\end{frame}

\begin{frame}[fragile]

  \begin{lstlisting}[language=rust]
fn main() {
    let mut x = 5;
    let r1 = &x;
    let r2 = &mut x;
    println!("{}", r1);
}
\end{lstlisting}

  \pause {\centering \xmark}

\end{frame}

\begin{frame}[fragile]

  \begin{lstlisting}[language=rust]


fn main() {
    let mut x = 5;
    {
        let r1 = &mut x;
        *r1 += 1;
    } 
    let r2 = &mut x;
    *r2 += 1;
    println!("{}", x);
}
\end{lstlisting}

  \pause {\centering \cmark}

  \pause On the other hand...

\end{frame}

\begin{frame}[fragile]

  \begin{lstlisting}[language=rust]
struct Boxed(i32);

impl Boxed {
  fn inc(&mut self) -> () {
    self.0 += 1;
  }
  fn consume(self) -> () {
    ()
  }
  fn inc_immutable(self) -> Self {
    Boxed(self.0 + 1)
  }
}

fn main() {
  let x = Boxed(5); // immutable struct created
  let mut x = x.inc_immutable(); // consumed and bound to a new mutable variable
  x.consume(); // moved and consumed
  x.inc(); // error: use after move
}
\end{lstlisting}

\end{frame}

%%%%%%%%%%%%%%%%%%%%%%%%%%%%%%%%%%%%%%%%%%
% that does it for Rust

\begin{frame}[fragile]
  \frametitle{What Verus is (and what it isn't)}

  Verus:

  \begin{itemize}[<+->]
    \item[\cmark] is a verifier for Rust programs,
    \item[\cmark] is a framework for writing specifications and proofs in Rust,
    \item[\cmark] provides SMT encoding for said code and proofs, and
    \item[\cmark] provides semantics for ghost code and its elimination,
  \end{itemize}

  \pause
  but:

  \begin{itemize}[<+->]
    \item[\xmark] isn't a verifier or model for unsafe Rust, and
    \item[\xmark] isn't even a model for safe Rust
  \end{itemize}

  \pause
  Verus builds on the existing type and borrow checking done by the Rust
  compiler. \pause A program is safe iff it passes the Rust compiler \emph{and}
  Verus.

\end{frame}

% verifying a simple program in Rust

\begin{frame}

  \begin{center}
    \large
    Writing a simple program with Verus + SMT encoding
  \end{center}

\end{frame}

% what the three modes are

\begin{frame}

  \centering

  \begin{tabular}{c | c c c}
                             & \lstinline|spec| & \lstinline|proof| & \lstinline|exec| \\
    \hline
    Call spec                & \cmark           & \cmark            & \cmark           \\
    Call proof               & \xmark           & \cmark            & \cmark           \\
    Call exec                & \xmark           & \xmark            & \cmark           \\
    \pause
    Compiled to machine code & \xmark           & \xmark            & \cmark           \\
    \pause
    Mutation                 & \xmark           & \cmark            & \cmark           \\
    \pause
    "SMT effects"            & \xmark           & \cmark            & \cmark           \\
    \pause
    Borrow-checking          & \xmark           & \cmark            & \cmark           \\
    \pause
    SMT Types                & \cmark           & \cmark            & \xmark           \\
    Pre/post-conditions      & \xmark           & \cmark            & \cmark           \\
  \end{tabular}
\end{frame}

% why do we need *three*?
%% spec is effectively functional, and good for writing, well, specifications
%% proof is the new middle ground, wherein it does still ahve to pass the borrow checker
%% and linearity lends itself to using capabilities, in a sense
%% this is where the "linear ghost types" part of the title comes in

% so, why do we need these things at all>

% mutexes in rust
%% does something look weird here?
%% why is this even allowed?
\begin{frame}[fragile]
  \frametitle{Mutexes in Rust}

  \begin{lstlisting}[language=rust]
const N: usize = 10;
let data = Arc::new(Mutex::new(0));

let (tx, rx) = channel();
for _ in 0..N {
    let (data, tx) = (Arc::clone(&data), tx.clone());
    thread::spawn(move || {
        let mut data = (* data).lock().unwrap();
        *data += 1;
        if *data == N {
            tx.send(()).unwrap();
        }
    });
}

rx.recv().unwrap();
\end{lstlisting}

\end{frame}

% interior mutability and the myth of safe rust
%% <mutex source code>

\begin{frame}[fragile]
  \frametitle{Interior Mutability and the Myth of Safe Rust}

  \begin{lstlisting}{language=rust}
impl<T> Mutex<T> {
  // ...

  pub fn lock(&self) -> LockResult<MutexGuard<'_, T>> {
    unsafe {
        self.inner.lock();
        MutexGuard::new(self)
    }
  }
}
  \end{lstlisting}

\end{frame}

% what does Verus do with this code?
%% nothing
%% Verus has no support for unsafe code

\begin{frame}
  \frametitle{Unsafe code in Verus}

  What does Verus do with this code? \pause Nothing.

  \pause
  Verus does not attempt to (directly) verify unsafe code.

\end{frame}

% permissions
%% instead, it provides a primitive for interior mutability
%% Verus mutable concurrent cell
\begin{frame}[fragile]
  \frametitle{Interior Mutability via Linear Ghost Permissions}

  Instead, it provides primitives that allow proving safety of code like the
  Mutex example before. \pause This is made possible in part by the fact that
  proof code is linearity checked.


  \begin{lstlisting}[language=rust]
    open_local_invariant!(&self.perm_inv => perm => {
      r = self.pcell.replace(&mut perm, val);
    });
  \end{lstlisting}

  \pause
  Once open, the permission is used like a capability, and cannot be copied or
  leaked due to borrow checking. Double-opening is guarded against by
  the verifier specifically.

  \pause
  These objects can be verified as destroyed by consuming the permission object
  (see PPtr).

\end{frame}

\begin{frame}[fragile]
  \frametitle{On (Local) Invariants}

  It is called an invariant because it additionally ensures that the value
  stored in the cell satisfies the type invariant:

  \begin{lstlisting}[language=rust]
impl<K, V, Pred: InvariantPredicate<K, V>> LocalInvariant<K, V, Pred>
  pub proof fn new(k: K, tracked v: V, ns: int) -> tracked i : LocalInvariant<K, V, Pred>
    requires
      Pred::inv(k, v),
    ensures
      i.constant() == k,
      i.namespace() == ns,
  \end{lstlisting}

  The invariant checks are in \lstinline|proof| mode and are erased.
  \lstinline|InvariantPredicate| is a typeclass just providing the
  \lstinline|inv| function.

\end{frame}

\begin{frame}
  
  \begin{center}
    {\Large Thanks! Questions?}

    If we have more time we'll discuss and look at how certain things are
    implemented in the (Rust or Verus) standard library or play with the
    verifier.
  \end{center}
  
\end{frame}

\end{document}
